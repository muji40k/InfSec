\leftsection{Аналитическая часть}

\vspace{-1\baselineskip}

% \begin{figure}[h]
%     \centering
%     \def\svgwidth{0.45\textwidth}
%     \input{aes.pdf_tex}
%     \caption{Алгоритм работы AES}
% \end{figure}

\subsection{Сжатие информации}

Сжатие информации --- процесс уменьшения объема данных с сохранением
возможности их полного восстановления.

При этом выделяют следующие классы сжатия информации:
\begin{enumerate}
    \item без потери информации;
    \item с потерей информации.
\end{enumerate}

Преимущество методов сжатия с потерями над методами сжатия без потерь состоит в
том, что первые делают возможной большую степень сжатия, продолжая
удовлетворять поставленным требованиям, а именно --- искажения должны быть в
допустимых пределах чувствительности человеческих органов, физических чувств.

\subsection{Алгоритм сжатия LZW}

LZW --- алгоритм сжатия без потерь. Основная идея метода заключается в замене
наиболее часто встречаемых последовательностей байт на более короткие кодовые
последовательности.

Кодовые последовательности получаются как двенадцатибитные индексы
соответствующих записей в таблице замен и потому могут использоваться повторно.

\subsubsection{Сжатие}

Как следует из описания, сжатое сообщение получается из исходного
последовательной заменой по таблице. Изначально таблица заполнена 256
возможными символами из таблицы ASCII.

На каждой итерации алгоритма ищется наибольшая подстрока, содержащаяся
в таблице. Соответствующее кодовая последовательность заносится в результат,
подстрока, состоящая из найденной и следующего символа добавляется в таблицу.

\subsubsection{Распаковка}

Преимуществом данного алгоритма является то, что таблица замены может быть
получена и в ходе распаковки, поэтому она не записывается в результирующее
сообщение. Это достигается за счет того, что аналогичные записи могут быть
построены конкатенацией предыдущей записи с первым символом найденной. При
этом возможна ситуация, когда подстрока не может быть заменена, что происходит,
когда замена должна производится индексом, полученным на текущей итерации,
поэтому значение формируется сложением предыдущей записи с ее первым символом.

