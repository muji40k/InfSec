\leftsection{Аналитическая часть}

\vspace{-1\baselineskip}

% \begin{figure}[h]
%     \centering
%     \def\svgwidth{0.45\textwidth}
%     \input{aes.pdf_tex}
%     \caption{Алгоритм работы AES}
% \end{figure}

\subsection{Электронная цифровая подпись}

Электронная цифровая подпись --- набор криптографических методов, позволяющий
подтвердить авторство электронного документа (будь то реальное лицо или,
например, аккаунт в криптовалютной системе).

Как правило, подпись получается при помощи шифрования документа с
использованием асимметричного шифрования закрытым ключом автора.

Для сокращения объема шифруемой и отправляемой информации подписываемая
информация первоначально хешируется.

\subsection{Алгоритм шифрования RSA}

RSA --- криптографический алгоритм с открытым ключом, основывающийся на
вычислительной сложности задачи факторизации больших чисел.

Задача заключается в нахождении таких чисел $e$, $d$ и $n$ таких, что для
любого $0 \le m < n$
\begin{equation*}
    m^{ed} = m \mod{n}.
\end{equation*}

Открытым ключом в данном алгоритме является пара $(e, n)$, закрытым ---
пара $(d, n)$.

Таким образом шифр текст получается в результате следующей операции
\begin{equation*}
    c = m^e \mod n,
\end{equation*}
расшифрованный текст
\begin{equation*}
    r = c^d \mod n.
\end{equation*}

\subsubsection{Алгоритм получения ключа}

RSA-ключи генерируются следующим образом:
\begin{enumerate}
    \item выбираются два различных случайных простых числа $p$ и $q$ заданного
          размера;
    \item вычисляется их произведение $n = p \cdot q$, которое называется модулем;
    \item вычисляется значение функции Эйлера от числа $n$:
          $\varphi(n) = (p-1) \cdot (q-1)$;
    \item выбирается целое число $1 < e <\varphi(n)$, взаимно простое
          со значением функции $\varphi(n)$; обычно в качестве e берут простые
          числа, содержащие небольшое количество единичных бит в двоичной
          записи, например, простые из чисел Ферма: 17, 257 или 65537;
    \item вычисляется число d, мультипликативно обратное к числу $e$ по модулю 
          $\varphi(n)$, то есть число, удовлетворяющее сравнению:
          $d \cdot e = 1 \mod {\varphi(n)}$ (обычно оно вычисляется при помощи
          расширенного алгоритма Евклида);
\end{enumerate}

\subsection{Алгоритм хеширования MD5}

MD5 --- 128-битный алгоритм хеширования. Предназначен для создания
<<отпечатков>> или дайджестов сообщения произвольной длины и последующей
проверки их подлинности.

Последовательность действий для алгоритма:
\begin{itemize}
    \item длина сообщения выравнивается до длины в 512 бит (даже если оно
          уже удовлетворяет этому условию) по следующим правилам
          \begin{itemize}
              \item сообщение дополняется единичным битом;
              \item сообщение дополняется нулевыми битами до длины,
                    сравнимой 448 по модулю 512;
              \item оставшиеся 64 бита заполняются соответствующим
                    представлением длины сообщения в битах до расширения;
          \end{itemize}
    \item инициализируются 4 32 битных буфера \code{A}, \code{B}, \code{C},
          \code{D} следующими значениями (и 4 соответствующих раундовых
          \code{a}, \code{b}, \code{c}, \code{d})
          \begin{align*}
            A & = 67452301h \\
            B & = EFCDAB89h \\
            C & = 98BADCFEh \\
            D & = 10325476h
          \end{align*}
    \item основной этап хеширования 4 этапа по 16 раундов
          \begin{itemize}
              \item текущие значения буферов заносятся в соответствующие
                    раундовые;
              \item значение буфера \code{a} вычисляется следующим образом
                    \begin{equation*}
                        a = b + ((a + \mathrm{F_i}(b, c, d) + T[i] + X[i]) <<< s[i]),
                    \end{equation*}
                    где $\mathrm{F_i}$ --- раундовая функция,
                    $T[i] = \lfloor 2 ^ 32 \cdot |\sin(i)| \rfloor$,
                    $X[i]$ --- текущее 32 битное слово сообщения,
                    $s[i]$ --- величина левого циклического сдвига;
              \item производится циклический сдвиг вправо;
              \item значение раундовых буферов складывается с результирующими;
          \end{itemize}
    \item полученные буферы записываются в порядке \code{ABCD}, порядок байт
          --- little-endian.
\end{itemize}

Раундовые функции
\begin{align*}
    \mathrm{F_0} & = \mathrm{F} = (b \land c) \lor (\neg b \land d) \\
    \mathrm{F_1} & = \mathrm{G} = (b \land d) \lor (\neg d \land c) \\
    \mathrm{F_2} & = \mathrm{H} = b \oplus c \oplus d \\
    \mathrm{F_3} & = \mathrm{I} = c \oplus (\neg d \lor b).
\end{align*}


