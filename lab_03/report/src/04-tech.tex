\leftsection{Технологическая часть}

\subsection{Описание программного обеспечения}

Для реализации машины использовался язык C++. Конфигурация осуществляется
при помощи файла config.json, расположенного в корневом каталоге.

\begin{figure}[!h]
    \begin{verbatim}
./config.json
{
    "initial_value": "./config/initial_vector",
    "aes": [
        {"key": "./config/keys/key1", "offset": 1},
        {"key": "./config/keys/key2", "offset": 2},
        {"key": "./config/keys/key3", "offset": 1}
    ]
}
    \end{verbatim}
\end{figure}

Соответсвтующие концигурационный  файлы состоят из последовательности байт:
16 для начального вектора; 16, 24 или 32 для ключа.

\begin{lstlisting}[language=c++, caption={Класс, реализующий алгоритм AES}]
void AES::encodeBlock(const unsigned char *in,
                      unsigned char *out)
{
    unsigned char current[4][4] = {{0}};
    unsigned char *currentp = (unsigned char *)current;
    const unsigned char *keyp;
    for (size_t i = 0; 4 > i; i++)
        for (size_t j = 0; 4 > j; j++)
            current[j][i] = in[4 * i + j];
    keyp = this->key_block->get(0);
    for (size_t i = 0; 16 > i; i++)
        currentp[i] ^= keyp[i];
    for (size_t iter = 2; this->iters > iter; iter++)
    {
        for (size_t i = 0; 16 > i; i++)
            currentp[i] = SBlock::direct(currentp[i]);
        RowShifter::direct(current);
        ColumnMixer::direct(current);
        keyp = this->key_block->get(iter - 1);
        for (size_t i = 0; 16 > i; i++)
            currentp[i] ^= keyp[i];
    }
    for (size_t i = 0; 16 > i; i++)
        currentp[i] = SBlock::direct(currentp[i]);
    RowShifter::direct(current);
    keyp = this->key_block->get(this->iters - 1);
    for (size_t i = 0; 16 > i; i++)
        currentp[i] ^= keyp[i];
    for (size_t i = 0; 4 > i; i++)
        for (size_t j = 0; 4 > j; j++)
            out[4 * i + j] = current[j][i];
}

void AES::decodeBlock(const unsigned char *in,
                      unsigned char *out)
{
    unsigned char current[4][4] = {{0}};
    unsigned char *currentp = (unsigned char *)current;
    const unsigned char *keyp;
    for (size_t i = 0; 4 > i; i++)
        for (size_t j = 0; 4 > j; j++)
            current[j][i] = in[4 * i + j];
    keyp = this->key_block->get(this->iters - 1);
    for (size_t i = 0; 16 > i; i++)
        currentp[i] ^= keyp[i];
    RowShifter::inverse(current);
    for (size_t i = 0; 16 > i; i++)
        currentp[i] = SBlock::inverse(currentp[i]);
    for (size_t iter = this->iters - 2; 0 < iter; iter--)
    {
        keyp = this->key_block->get(iter);
        for (size_t i = 0; 16 > i; i++)
            currentp[i] ^= keyp[i];
        ColumnMixer::inverse(current);
        RowShifter::inverse(current);
        for (size_t i = 0; 16 > i; i++)
            currentp[i] = SBlock::inverse(currentp[i]);
    }
    keyp = this->key_block->get(0);
    for (size_t i = 0; 16 > i; i++)
        currentp[i] ^= keyp[i];
    for (size_t i = 0; 4 > i; i++)
        for (size_t j = 0; 4 > j; j++)
            out[4 * i + j] = current[j][i];
}
\end{lstlisting}

\begin{lstlisting}[language=c++, caption={Класс, реализующий получение
                                          раундовых ключей}]
KeyBlock::KeyBlock(const std::string &key)
{
    this->checkKey(key);
    this->keys = \
    std::make_unique<unsigned char[]>(this->niters * 16);
    memmove(this->keys.get(), key.data(), this->length);
    unsigned char rc = 1;

    for (size_t block = this->length / 4,
                blocks = block,
                limit = this->niters * 4;
         limit > block; block++)
    {
        unsigned char *current  = &this->keys[4 * block],
                      *previous = current - this->length;
        this->blockCopy(current, current - 4);

        if (0 == block % blocks)
            this->blockRc(
                this->blockXOR(
                    this->blockS(this->blockShift(current)),
                    previous),
                &rc);
        else if (32 == length && 4 == block % blocks)
            this->blockXOR(this->blockS(current), previous);
        else
            this->blockXOR(current, previous);
    }
}
\end{lstlisting}

\subsection{Тестирование}
\begin{table}[h]
\small
\begin{tabular}{|r|l|l|l|}
\hline
№ & Исходные данные                                                                                                            & Ожидаемый результат                                                                                                                                                                           & Фактический результат                                                                                                                                                                                                                                                                                                                                                                                                                                                                                                                \\ \hline
1 & \begin{tabular}[c]{@{}l@{}}abcdefgh\\ ijklmnop\\ qrstuvwx\\ yz\textbackslash{}0a\end{tabular}                              & \begin{tabular}[c]{@{}l@{}}abcdefgh\\ ijklmnop\\ qrstuvwx\\ yz\textbackslash{}x0a\textbackslash{}x00\textbackslash{}x00\textbackslash{}x00\textbackslash{}x00\textbackslash{}x00\end{tabular} & \begin{tabular}[c]{@{}l@{}}abcdefgh\\ ijklmnop\\ qrstuvwx\\ yz\textbackslash{}x0a\textbackslash{}x00\textbackslash{}x00\textbackslash{}x00\textbackslash{}x00\textbackslash{}x00\end{tabular}                                                                                                                                                                                                                                                                                                                                        \\ \hline
2 & Пустой файл                                                                                                                & Пустой файл                                                                                                                                                                                   & Пустой файл                                                                                                                                                                                                                                                                                                                                                                                                                                                                                                                          \\ \hline
3 & \begin{tabular}[c]{@{}l@{}}Попытка расшифровать\\ сообщение из теста №1\\ с другим ключом\\ (во втором блоке)\end{tabular} & \begin{tabular}[c]{@{}l@{}}abcdefghijklmnop \\ \textless{}блок 16 байт\textgreater{}\end{tabular}                                                                                  & \begin{tabular}[c]{@{}l@{}}abcdefghijklmnop\\ \textbackslash{}xac\textbackslash{}x68\textbackslash{}x64\textbackslash{}xbe \\ \textbackslash{}x8e\textbackslash{}xae\textbackslash{}x82\textbackslash{}x55 \\ \textbackslash{}x8c\textbackslash{}x7a\textbackslash{}x45\textbackslash{}xcc \\ \textbackslash{}xd4\textbackslash{}x8c\textbackslash{}x6a\textbackslash{}x42 \end{tabular} \\ \hline
\end{tabular}
\end{table}



