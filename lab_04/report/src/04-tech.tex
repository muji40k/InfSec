\leftsection{Технологическая часть}

\subsection{Описание программного обеспечения}

Для реализации машины использовался язык C++. Конфигурация осуществляется
при помощи файлов private.json и public.json, расположенных в корневом каталоге,
содержащих соответствующие значения закрытого и открытого ключей.
Длина ключа выбрана равной 64 бита.

\begin{figure}[!h]
    \begin{verbatim}
./public.json
{
    "e": 17,
    "n": 479653117661354309
}
./private.json
{
    "d": 282148891917378473,
    "n": 479653117661354309
}
    \end{verbatim}
\end{figure}

\begin{lstlisting}[language=c++, caption={Класс, реализующий алгоритм RSA}]
class RSA : public Encoder
{
    public:
        RSA(unsigned long long n);
        void setPublicKey(unsigned long long e);
        void setPrivateKey(unsigned long long d);
        virtual ~RSA(void) override = default;
        virtual std::string encode(const std::string &origin) override;
        virtual std::string decode(const std::string &origin) override;
    private:
        unsigned long long e;
        unsigned long long d;
        const unsigned long long n;
};

static unsigned long long modadd(unsigned long long a,
                                 unsigned long long b,
                                 unsigned long long mod);
static unsigned long long modmul(unsigned long long a,
                                 unsigned long long b,
                                 unsigned long long mod);
static unsigned long long modpow(unsigned long long a,
                                 unsigned long long pow,
                                 unsigned long long mod);


std::string RSA::encode(const std::string &origin)
{
    if (0 == this->e)
        throw;
    size_t size = origin.length(), rest = size % 7;
    size = size / 7 + (rest ? 1 : 0);
    std::string out;
    out.reserve(size * 8);
    unsigned long long current, mask = 0x00ffffffffffffff;
    for (size_t i = 0; size > i; i++)
    {
        if (size == i + 1 && rest)
        {
            mask = 0;
            for (size_t j = 0; rest > j; j++)
                mask = (mask << 8) | 0xff;
        }
        current = *(unsigned long long *)(origin.data() + 7 * i)
                  & mask;
        current = modpow(current, this->e, this->n);
        out += std::string((char *)&current, 8);
    }
    return out;
}

std::string RSA::decode(const std::string &origin)
{
    if (0 == this->d)
        throw;
    size_t size = origin.length(), psize = size / 8;
    if (0 != size % 8)
        throw;
    std::string out;
    out.reserve(size - psize);
    out.resize(size - psize);
    const unsigned long long *inp =
    (unsigned long long *)origin.data();
    unsigned long long current;
    char *outp = out.data(), *currentp = (char *)&current;
    for (size_t i = 0; psize > i; i++)
    {
        current = inp[i];
        current = modpow(current, this->d, this->n);
        for (size_t j = 0; 7 > j; j++)
            *(outp++) = currentp[j];
    }
    return out;
}

static unsigned long long modadd(unsigned long long a,
                                 unsigned long long b,
                                 unsigned long long mod)
{
    a %= mod;
    b %= mod;
    unsigned long long diff = mod - b;
    if (a >= diff)
        return a - diff;
    return a + b;
}

static unsigned long long modmul2(unsigned long long a,
                                  unsigned long long mod)
{
    unsigned long long shrinked = mod >> 1, rest = mod & 1;
    if (a >= shrinked)
        return ((a - shrinked) << 1) - rest;
    return a << 1;
}

static unsigned long long modmul(unsigned long long a,
                                 unsigned long long b,
                                 unsigned long long mod)
{
    a %= mod;
    b %= mod;
    unsigned long long result = 0;
    for (; b; b >>= 1, a = modmul2(a, mod))
        if (b & 1)
            result = modadd(result, a, mod);
    return result;
}

static unsigned long long modpow(unsigned long long a,
                                 unsigned long long pow,
                                 unsigned long long mod)
{
    a %= mod;
    unsigned long long result = 1;
    for (; pow; pow >>= 1)
    {
        if (pow & 1)
            result = modmul(result, a, mod);
        a = modmul(a, a, mod);
    }
    return result;
}
\end{lstlisting}

\begin{lstlisting}[language=c++, caption={Класс, реализующий алгоритм MD5}]
class MD5 : public HashFunction
{
    public:
        virtual ~MD5(void) override = default;
        virtual std::string apply(const std::string &origin) override;
    private:
        using FFunc = unsigned int (*)(const unsigned int,
                                       const unsigned int,
                                       const unsigned int);
        using Iter = struct
        {
            FFunc func;
            size_t k;
        };
        static void getIteration(const size_t i, Iter &handler);
        static unsigned int F(const unsigned int b,
                              const unsigned int c,
                              const unsigned int d);
        static unsigned int G(const unsigned int b,
                              const unsigned int c,
                              const unsigned int d);
        static unsigned int H(const unsigned int b,
                              const unsigned int c,
                              const unsigned int d);
        static unsigned int I(const unsigned int b,
                              const unsigned int c,
                              const unsigned int d);
    private:
        static const unsigned char expansion[64];
        static const size_t S[64];
        static const unsigned int K[64];
};

const unsigned char MD5::expansion[64] = { 0x80 };

const size_t MD5::S[64] =
{
    ...
};

const unsigned int MD5::K[64] =
{
    ...
};

template <typename Type>
Type shift_left(const Type val, size_t size)
{
    static constexpr const size_t tsize = sizeof(Type) << 3;
    size %= tsize;
    return (val << size) | (val >> (tsize - size));
}

std::string MD5::apply(const std::string &origin)
{
    size_t size = origin.length(), esize;
    const size_t bits = size * 8;
    const size_t rest = size % 64;
    if (56 > rest)
        esize = 56 - rest;
    else
        esize = 120 - rest;
    size += esize + 8;
    std::string copy;
    copy.reserve(size);
    copy += origin + std::string((const char *)MD5::expansion, esize)
            + std::string((char *)&bits, 8);
    const unsigned char *data = (unsigned char *)copy.data();
    const unsigned int *current = nullptr;
    unsigned int A = 0x67452301, a,
                 B = 0xEFCDAB89, b,
                 C = 0x98BADCFE, c,
                 D = 0x10325476, d,
                 F = 0;
    Iter handler;
    for (size_t i = 0, blocks = size >> 6; blocks > i; i++)
    {
        a = A, b = B, c = C, d = D;
        current = (const unsigned int *)(data + i * 64);
        for (size_t j = 0; 64 > j; j++)
        {
            MD5::getIteration(j, handler);

            F = handler.func(b, c, d) + a + MD5::K[j]
                + current[handler.k];
            a = d;
            d = c;
            c = b;
            b += shift_left<unsigned int>(F, MD5::S[j]);
        }
        A += a, B += b, C += c, D += d;
    }
    return std::string((char *)&A, 4) + std::string((char *)&B, 4)
           + std::string((char *)&C, 4) + std::string((char *)&D, 4);
}

void MD5::getIteration(const size_t i, Iter &handler)
{
    if (16 > i)
    {
        handler.func = &MD5::F;
        handler.k = i;
    }
    else if (32 > i)
    {
        handler.func = &MD5::G;
        handler.k = (5 * i + 1) % 16;
    }
    else if (48 > i)
    {
        handler.func = &MD5::H;
        handler.k = (3 * i + 5) % 16;
    }
    else
    {
        handler.func = &MD5::I;
        handler.k = (7 * i) % 16;
    }
}

unsigned int MD5::F(const unsigned int b, const unsigned int c,
                    const unsigned int d)
{
    return (b & c) | (~b & d);
}

unsigned int MD5::G(const unsigned int b, const unsigned int c,
                    const unsigned int d)
{
    return (b & d) | (~d & c);
}

unsigned int MD5::H(const unsigned int b, const unsigned int c,
                    const unsigned int d)
{
    return b ^ c ^ d;
}

unsigned int MD5::I(const unsigned int b, const unsigned int c,
                    const unsigned int d)
{
    return c ^ (~d | b);
}
\end{lstlisting}

\subsection{Тестирование}

\vspace{-2\baselineskip}

\begin{table}[h]
\footnotesize
\caption{Тестирование алгоритма MD5}
\begin{center}
\begin{tabular}{|r|l|l|l|}
\hline
№ & Исходные данные                                                                        & Ожидаемый результат              & Фактический результат            \\ \hline
1 & \textless{}Пустая строка\textgreater{}                                                 & d41d8cd98f00b204e9800998ecf8427e & d41d8cd98f00b204e9800998ecf8427e \\ \hline
2 & \begin{tabular}[c]{@{}l@{}}abcdefghijklm\\ nopqstuvwxyz\textbackslash{}0a\end{tabular} & e302f9ecd2d189fa80aac1c3392e9b9c & e302f9ecd2d189fa80aac1c3392e9b9c \\ \hline
3 & md5                                                                                    & 1bc29b36f623ba82aaf6724fd3b16718 & 1bc29b36f623ba82aaf6724fd3b16718 \\ \hline
\end{tabular}
\end{center}
\end{table}

Для тестирования алгоритма RSA использовались ключи, приведенные выше.

\begin{table}[h]
\small
\caption{Тестирование алгоритма RSA (шифрование)}
\begin{center}
\begin{tabular}{|r|r|r|r|}
\hline
№ & Исходные данные    & Ожидаемый результат & Фактический результат \\ \hline
1 & 1                  & 1                   & 1                     \\ \hline
2 & 123                & 159162881495752659  & 159162881495752659    \\ \hline
3 & 324440504654685957 & 64871042383186893   & 64871042383186893     \\ \hline
\end{tabular}
\end{center}
\end{table}

\begin{table}[h]
\small
\caption{Тестирование алгоритма RSA (расшифровка)}
\begin{center}
\begin{tabular}{|r|r|r|r|}
\hline
№ & Исходные данные    & Ожидаемый результат & Фактический результат \\ \hline
1 & 1                  & 1                   & 1                     \\ \hline
2 & 159162881495752659 & 123                 & 123                   \\ \hline
3 & 64871042383186893  & 324440504654685957  & 324440504654685957    \\ \hline
\end{tabular}
\end{center}
\end{table}

