\leftsection{Технологическая часть}

\subsection{Описание программного обеспечения}

Для реализации машины использовался язык C++.

\begin{lstlisting}[language=c++, caption={Метод, реализующий сжатие}]
std::string LZW::compress(const std::string &origin)
{
    if (origin.empty())
        return origin;

    std::string out;
    WTable table;
    BufferWriter writer(out);
    std::string sequence = "", last = "";

    for (char c : origin)
    {
        last = sequence;
        sequence += c;

        if (!table.present(sequence))
        {
            writer.add(table.get(last));
            table.add(sequence);
            sequence = c, last = "";
        }
    }

    writer.add(table.get(sequence));
    writer.flush();

    return out;
}
\end{lstlisting}

\clearpage

\begin{lstlisting}[language=c++, caption={Метод, реализующий распаковку}]
std::string LZW::decompress(const std::string &origin)
{
    if (origin.empty())
        return origin;

    std::string out;
    RTable table;
    BufferReader reader(origin);
    unsigned short code = reader.read();
    std::string previous, current;

    previous = table.get(code);
    out += previous;

    while (reader.rest())
    {
        code = reader.read();

        if (table.present(code))
        {
            current = table.get(code);
            table.add(previous + current[0]);
        }
        else // Handle mirror sequence c.*c.*c (skip of index)
        {
            current = previous + previous[0];
            table.add(current);
        }

        out += current;
        previous = current;
    }

    return out;
}
\end{lstlisting}

\clearpage

\subsection{Тестирование}

\vspace{-2\baselineskip}

\begin{table}[h]
\caption{Тестирование алгоритма сжатия}
\begin{center}
\begin{tabular}{|r|l|l|l|}
\hline
№ & Исходные данные                                                            & Ожидаемый результат                                                            & Фактический результат                                                          \\ \hline
1 & Пустой файл                                                                & Пустой файл                                                                    & Пустой файл                                                                    \\ \hline
2 & <31>                                                                       & <31><00>                                                                       & <31><00>                                                                       \\ \hline
3 & \begin{tabular}[c]{@{}l@{}}<31><31><31>\\ <31><31><31>\\ <31>\end{tabular} & \begin{tabular}[c]{@{}l@{}}<31><00><10>\\ <01><11><03>\end{tabular}            & \begin{tabular}[c]{@{}l@{}}<31><00><10>\\ <01><11><03>\end{tabular}            \\ \hline
4 & <61><62><63>                                                               & \begin{tabular}[c]{@{}l@{}}<61><20><06>\\ <63><00>\end{tabular}                & \begin{tabular}[c]{@{}l@{}}<61><20><06>\\ <63><00>\end{tabular}                \\ \hline
5 & \begin{tabular}[c]{@{}l@{}}<61><62><63>\\ <61><62><63>\end{tabular}        & \begin{tabular}[c]{@{}l@{}}<61><20><06>\\ <63><00><10>\\ <63><00>\end{tabular} & \begin{tabular}[c]{@{}l@{}}<61><20><06>\\ <63><00><10>\\ <63><00>\end{tabular} \\ \hline
\end{tabular}
\end{center}
\end{table}

\begin{table}[h]
\caption{Тестирование алгоритма распаковки}
\begin{center}
\begin{tabular}{|r|l|l|l|}
\hline
№ & Исходные данные                                                                & Ожидаемый результат                                                        & Фактический результат                                                      \\ \hline
1 & Пустой файл                                                                    & Пустой файл                                                                & Пустой файл                                                                \\ \hline
2 & <31><00>                                                                       & <31>                                                                       & <31>                                                                       \\ \hline
3 & \begin{tabular}[c]{@{}l@{}}<31><00><10>\\ <01><11><03>\end{tabular}            & \begin{tabular}[c]{@{}l@{}}<31><31><31>\\ <31><31><31>\\ <31>\end{tabular} & \begin{tabular}[c]{@{}l@{}}<31><31><31>\\ <31><31><31>\\ <31>\end{tabular} \\ \hline
4 & \begin{tabular}[c]{@{}l@{}}<61><20><06>\\ <63><00>\end{tabular}                & <61><62><63>                                                               & <61><62><63>                                                               \\ \hline
5 & \begin{tabular}[c]{@{}l@{}}<61><20><06>\\ <63><00><10>\\ <63><00>\end{tabular} & \begin{tabular}[c]{@{}l@{}}<61><62><63>\\ <61><62><63>\end{tabular}        & \begin{tabular}[c]{@{}l@{}}<61><62><63>\\ <61><62><63>\end{tabular}        \\ \hline
6 & <31>                                                                           & Ошибка                                                                     & Ошибка                                                                     \\ \hline
\end{tabular}
\end{center}
\end{table}

