\leftsection{Аналитическая часть}

\vspace{-1\baselineskip}

\subsection{Описание алгоритма}

AES (Rijndael) ---  симметричный алгоритм блочного шифрования (размер блока 128
бит, ключ 128/192/256 бит). Алгоритм шифрования базируется на применении
SP-сетей, что представляют из себя последовательность применения блок
замены (S) и перемешивания (P) к блоку открытого для получения блока шифр
текста. Основной вычислений являются вычисления в конечном поле Галуа
$\mathrm{GF}(2^8)$.

Изначально блок сообщения преобразуется для в матрицу, элементы которой
располагается по столбцам
\begin{equation*}
    \begin{bmatrix}
        b_0 & b_4 & b_8  & b_{12} \\
        b_1 & b_5 & b_9  & b_{13} \\
        b_2 & b_6 & b_{10} & b_{14} \\
        b_3 & b_7 & b_{11} & b_{15}
    \end{bmatrix}
\end{equation*}

Далее преобразования производятся по следующей схеме.

\begin{figure}[h]
    \centering
    \def\svgwidth{0.45\textwidth}
    \input{aes.pdf_tex}
    \caption{Алгоритм работы AES}
\end{figure}

\clearpage

\subsubsection{Получение раундовых ключей}

Раундовые ключи $K'_0$--$K'_n$ получаются при помощи расширения. Пример
расчета для 128 битного ключа приведен на рисунке.

\begin{figure}[h]
    \centering
    \def\svgwidth{0.6\textwidth}
    \input{k_main.pdf_tex}
    \caption{Алгоритм получения ключа}
\end{figure}

\begin{figure}[h]
    \centering
    \def\svgwidth{0.6\textwidth}
    \input{rc.pdf_tex}
    \caption{Алгоритм функции $\mathrm{g}$}
\end{figure}

$K_1$--$K_4$ --- байты исходного ключа, $rc$ -- раундовое значение из
поля Галуа $\mathrm{GF}(2^8)$, определяемое следующим соотношением
$rc_i = 2 \cdot rc_{i - 1}$.

\subsubsection{Блока замены}

Входной байт интерпретируется как многочлен поля Галуа, где значение
соответствующего разряда --- значение коэффициента при соответствующей
степени многочлена. Последовательность действий:
\begin{enumerate}
    \item нахождение обратного элемента;
    \item умножение вектора на постоянную матрицу;
    \item сумма с постоянным смещением $0x63$.
\end{enumerate}

\begin{equation*}
    \begin{bmatrix}
        1 & 0 & 0 & 0 & 1 & 1 & 1 & 1 \\
        1 & 1 & 0 & 0 & 0 & 1 & 1 & 1 \\
        1 & 1 & 1 & 0 & 0 & 0 & 1 & 1 \\
        1 & 1 & 1 & 1 & 0 & 0 & 0 & 1 \\
        1 & 1 & 1 & 1 & 1 & 0 & 0 & 0 \\
        0 & 1 & 1 & 1 & 1 & 1 & 0 & 0 \\
        0 & 0 & 1 & 1 & 1 & 1 & 1 & 0 \\
        0 & 0 & 0 & 1 & 1 & 1 & 1 & 1 \\
    \end{bmatrix}
    \cdot
    \begin{bmatrix}
        b_1 \\
        b_2 \\
        b_3 \\
        b_4 \\
        b_5 \\
        b_6 \\
        b_7 \\
        b_8
    \end{bmatrix}
    +
    \begin{bmatrix}
        1 \\
        1 \\
        0 \\
        0 \\
        0 \\
        1 \\
        1 \\
        0
    \end{bmatrix}
    =
    \begin{bmatrix}
        c_1 \\
        c_2 \\
        c_3 \\
        c_4 \\
        c_5 \\
        c_6 \\
        c_7 \\
        c_8
    \end{bmatrix}
\end{equation*}

Данные значение могут быть вычислены заранее для всех 255 элементов поля
и записаны в соответствующую таблицу. 

Для обратного преобразования необходимо обратная операция, которая определяется
следующим образом.

\begin{equation*}
    \begin{bmatrix}
        0 & 0 & 1 & 0 & 0 & 1 & 0 & 1 \\
        1 & 0 & 0 & 1 & 0 & 0 & 1 & 0 \\
        0 & 1 & 0 & 0 & 1 & 0 & 0 & 1 \\
        1 & 0 & 1 & 0 & 0 & 1 & 0 & 0 \\
        0 & 1 & 0 & 1 & 0 & 0 & 1 & 0 \\
        0 & 0 & 1 & 0 & 1 & 0 & 0 & 1 \\
        1 & 0 & 0 & 1 & 0 & 1 & 0 & 0 \\
        0 & 1 & 0 & 0 & 1 & 0 & 1 & 0 \\
    \end{bmatrix}
    \cdot
    \begin{bmatrix}
        c_1 \\
        c_2 \\
        c_3 \\
        c_4 \\
        c_5 \\
        c_6 \\
        c_7 \\
        c_8
    \end{bmatrix}
    +
    \begin{bmatrix}
        1 \\
        0 \\
        1 \\
        0 \\
        0 \\
        0 \\
        0 \\
        0
    \end{bmatrix}
    =
    \begin{bmatrix}
        b_1 \\
        b_2 \\
        b_3 \\
        b_4 \\
        b_5 \\
        b_6 \\
        b_7 \\
        b_8
    \end{bmatrix}
\end{equation*}

\subsubsection{Блоки перестановки}

Преобразование строк происходит левым циклическим сдвигом байт матрицы
сообщение в зависимости от номера строки.

Перестановка столбцов происходит умножением вектора-столбца на многочлен
$\begin{bmatrix} 2 \\ 1 \\ 1 \\ 3 \end{bmatrix}$. Обратное преобразование ---
умножением на многочлен $\begin{bmatrix} 14 \\ 9 \\ 13 \\ 11 \end{bmatrix}$.

\subsection{Режим шифрования CFB}
Режим шифрования --- метод применения блочного шифра, позволяющий преобразовать
последовательность блоков открытых данных в последовательность блоков
зашифрованных данных.

В данном режиме для шифрования следующего блока открытого текста происходит
сложение по модулю 2 с зашифрованным результатом шифрования предыдущего блока.

\begin{figure}[h]
    \centering
    \def\svgwidth{\textwidth}
    \input{cfb.pdf_tex}
    \caption{Схема режима шифрования CFB}
    \label{fig:pcbc}
\end{figure}

В результате, ошибка при передаче зашифрованного сообщения отразится только
при расшифровке поврежденного и следующего за ним блоков.

