\leftsection{Аналитическая часть}

\vspace{-1\baselineskip}

\subsection{Описание машины}

Энигма --- переносная шифровальная машина, использовавшаяся для шифрования и
расшифрования секретных сообщений.

Первая версия машины была разработана в 1918 году немецким инженером
Артуром Шербиусом. Данное устройство активно использовалась в военной сфере,
чтобы солдаты и командиры могли обмениваться конфиденциальной информацией.

\subsection{Составные части}

Энигма состояла из комбинации механических и электрических систем.
Механическая часть включала в себя клавиатуру, набор вращающихся
дисков --- роторов, которые были расположены вдоль вала и прилегали к нему,
ступенчатого механизма, двигающего один или несколько роторов при каждом
нажатии на клавишу, и рефлектора (набор попарно соединенных контактов,
отражающих сигнал в обратном направлении).

Электрическая часть, в свою очередь, состояла из электрической схемы,
соединяющей между собой клавиатуру, коммутационную панель, лампочки и роторы
(для соединения роторов, первичного вала и отразателя использовались скользящие
контакты).

\subsection{Принцип работы}

Текст, который нужно было зашифровать, печатался на машине с использованием
клавиатуры. Перед началом использования оператор открывал крышку аппарата и
устанавливал начальную позицию --- три номера, последовательность которых
передавалась заранее.

С каждым нажатием на клавишу производится смещение первого ротора на одну
ступень, при полном обороте текущего ротора также происходит сдвиг
следующего и так далее.

Электронный сигнал поступает от клавиатуры через коммутационную панель,
производящую первичное перемешивание пар сигналов. Далее сигнал поступает на
первичный ротор после чего проходит по текущей конфигурации роторов. Дойдя до
отражателя сигнал <<разворачивается>> и проходит через систему роторов еще
раз. Полученный сигнал отображается на индикационной панели, после обработки
на коммутационной панели.

Использование отражателя в системе позволяет кодировать и расшифровывать
сообщения на одной машине. Все что необходимо знать: первоначальную настройку
роторов и соединений на коммутационной панели.

