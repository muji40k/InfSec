\leftsection{Аналитическая часть}

\vspace{-1\baselineskip}

\subsection{Описание алгоритма}

DES (Data Encryption Standard) --- алгоритм для симметричного шифрования,
разработанный фирмой IBM и утверждённый правительством США в 1977 году как
официальный стандарт.

Алгоритм является блочным и основывается на применении сетей Фейстеля.
Открытый текст разбивается на блоки размером 64 бита, в случае необходимости
последний блок дополняется нулями. Применяемый ключ также имеет размер 64
бита, однако каждый 8 бит является битом четности и не участвует в шифровании
(фактический размер 56 бит).

Шифрование каждого блока состоит из следующих шагов:
\begin{itemize}
    \item производится первоначальная перестановка бит сообщения (блок $IP$);
    \item полученный результат разбивается на 2 блока по 32 бита ($L_0$ и $R_0$
          соответственно;
    \item выполняется побитовый $\mathrm{xor}$ левой части с результатом работы
          функции Фейстеля для ключа и правой части;
    \item результат предыдущего шага записывается в правую ячейку, правая же
          ячейка попадает в левую без изменений;
    \item 2 предыдущих шага (раунд) выполняются 16 раз
    \begin{align*}
        L_{i+1} & = R_i \\
        R_{i+1} & = L_i \oplus \mathrm{f}(R_i, k_i);
    \end{align*}
    \item применяется перестановка, обратная первоначальной ($IP^-1$).
\end{itemize}
Расшифровка происходит в обратном порядке.

Функция Фейстеля состоит из следующих этапов:
\begin{itemize}
    \item часть сообщения расширяется до 48 бит, за счет дополнения частей
          по 4 бита примыкающими битами;
    \item выполняется побитовый $\mathrm{xor}$ расширенного сообщения с
          i-ым ключом;
    \item полученный результат разделяется на 8 частей и поступает на вход
          соответствующим $S$-блокам, возвращающим 4 бита;
    \item фрагменты, полученный на предыдущем этапе, последовательно
          соединяются в итоговый результат.
\end{itemize}

$S$-блок является таблицей, состоящей из 4 строк и 16 столбцов, 
содержащей уникальные 4 битовые значения в пределах одной строки. Навигация по
таблице происходит по следующим правилам:
\begin{itemize}
    \item первый и последний биты определяют номер строки;
    \item средние --- номер столбца.
\end{itemize}

Ключ для каждого раунда вырабатывается следующим образом:
\begin{itemize}
    \item ключ разбивается на 2 половины согласно таблице;
    \item каждая половина циклически смещается влево, согласно значению
          таблицы на i-ом шаге;
    \item используя таблицу $H$, из половин составляется ключ для i-ого раунда
          шифрования.
\end{itemize}

\subsection{Режим шифрования PCBC}
Режим шифрования --- метод применения блочного шифра, позволяющий преобразовать
последовательность блоков открытых данных в последовательность блоков
зашифрованных данных.

Идея режима PCBC заключается в последовательном применении алгоритма
DES с различными ключами к соответствующим блокам открытого текста, при этом
перед шифровкой производится побитовый $\mathrm{xor}$ трех блоков:
предыдущего блока открытого текста, соответствующего ему блока шифр текста
и текущего блока открытого текста. Для первого блока используется значение
вектора инициализации.

\clearpage

\begin{figure}[h]
    \centering
    \def\svgwidth{\textwidth}
    \input{pcbc.pdf_tex}
    \caption{Схема режима шифрования PCBC}
    \label{fig:pcbc}
\end{figure}

Применение данного режима позволяет избежать статистических зависимостей
при применении алгоритма DES. Отличительной особенностью данного
метода является следующее свойство: если в каком-либо блоке шифр текста
присутствует ошибка, то все последующие блоки также будут расшифрованы
с ошибкой.

