\leftsection{Технологическая часть}

\subsection{Описание программного обеспечения}

Для реализации машины использовался язык C++. Конфигурация осуществляется
при помощи файла config.json, расположенного в корневом каталоге.

\begin{figure}[!h]
    \begin{verbatim}
./config.json
{
    "initial_value": "./config/initial_value",
    "des_blocks": [
        "./config/des1/config.json",
        "./config/des2/config.json",
        "./config/des3/config.json"
    ]
}
    \end{verbatim}
\end{figure}

\clearpage

\begin{figure}[!h]
    \begin{verbatim}
./config/des1/config.json
{
    "f_block": {
        "path": "./f_block",
        "s_blocks": [
            "./s1", "./s2", "./s3", "./s4",
            "./s5", "./s6", "./s7", "./s8"
        ],
        "p_block": "./p_block"
    },

    "key_block": {
        "path": "./key_block",
        "key": "./key",
        "p_block": {
            "c": "./c_block",
            "d": "./d_block"
        },
        "shift_blocks": "./shift_blocks",
        "h_block": "./h_block"
    },

    "ip_block": "./ip_block"
}
    \end{verbatim}
    \caption{Пример конфигурации системы}
\end{figure}

Соответсвтующие концигурационный файлы состоят из массива типа \verb|size_t|,
за исключением начального вектора и ключа каждого отдельного блока DES.

\clearpage

\begin{lstlisting}[language=c++, caption={Класс, реализующий алгоритм DES}]
DES::DES(std::shared_ptr<IPBlock> ip_block,
         std::shared_ptr<FBlock> f_block,
         std::shared_ptr<KeyBlock> key_block)
    : ip_block(ip_block), f_block(f_block),
      key_block(key_block)
{
    if (nullptr == ip_block || nullptr == f_block
        || nullptr == key_block)
        throw;
}

std::string DES::encode(const std::string &origin)
{
    size_t length = origin.length();

    if (0 == length)
        return origin;

    size_t rest = length % 8;

    if (0 != rest)
        length += 8 - rest;

    std::string out (length, 0);
    char *buffer = out.data();

    memmove(buffer, origin.c_str(), origin.length());

    BitRange main_rng (buffer, 0, length * 8);

    auto msgs = main_rng.splitSize(64);

    for (size_t i = 0; msgs.size() > i; i++)
        this->encodeBlock(msgs[i]);

    return out;
}



std::string DES::decode(const std::string &origin)
{
    size_t length = origin.length();

    if (0 == length)
        return origin;

    if (0 != length % 8)
        throw;

    std::string out (length, 0);
    char *buffer = out.data();

    memmove(buffer, origin.c_str(), length);

    BitRange main_rng (buffer, 0, length * 8);

    auto msgs = main_rng.splitSize(64);

    for (size_t i = 0; msgs.size() > i; i++)
        this->decodeBlock(msgs[i]);

    return out;
}

void DES::encodeBlock(BitRange current)
{
    if (64 != current.size())
        throw;

    char buffer[8], key_buffer[6];
    BitRange buf_rng (buffer, 0, 64),
             key_rng (key_buffer, 0, 48),
             buf_32(buffer, 0, 32);

    this->ip_block->direct(current, buf_rng);
    current.copy(buf_rng);
    auto splits = current.split(2);
    BitRange l = splits[0], r = splits[1];

    for (size_t i = 0; 16 > i; i++)
    {
        this->key_block->get(i, key_rng);
        this->f_block->apply(r, key_rng, buf_32);
        buf_32 ^= l;
        l.copy(r);
        r.copy(buf_32);
    }

    this->ip_block->reverse(current, buf_rng);
    current.copy(buf_rng);
}

void DES::decodeBlock(BitRange current)
{
    if (64 != current.size())
        throw;

    char buffer[8], key_buffer[6];
    BitRange buf_rng (buffer, 0, 64),
             key_rng (key_buffer, 0, 48),
             buf_32(buffer, 0, 32);

    this->ip_block->direct(current, buf_rng);
    current.copy(buf_rng);
    auto splits = current.split(2);
    BitRange l = splits[0], r = splits[1];

    for (size_t i = 0; 16 > i; i++)
    {
        this->key_block->get(15 - i, key_rng);
        this->f_block->apply(l, key_rng, buf_32);
        buf_32 ^= r;
        r.copy(l);
        l.copy(buf_32);
    }

    this->ip_block->reverse(current, buf_rng);
    current.copy(buf_rng);
}
\end{lstlisting}
\clearpage
\begin{lstlisting}[language=c++, caption={Класс, реализующий функцию Фейстеля}]
FBlock::FBlock(std::shared_ptr<EBlock> e_block,
               std::vector<std::shared_ptr<SBlock>> s_blocks,
               std::shared_ptr<FPBlock> p_block)
    : e_block(e_block), s_blocks(s_blocks), p_block(p_block)
{
    if (nullptr == e_block || nullptr == p_block)
        throw;

    for (auto block : s_blocks)
        if (nullptr == block)
            throw;
}

void FBlock::apply(BitRange in, BitRange key, BitRange out)
{
    if (32 != in.size() || 48 != key.size()
        || 32 != out.size())
        throw;

    char buffer[6], out_buffer[4];
    BitRange buf_rng (buffer, 0, 48),
             out_rng (out_buffer, 0, 32);
    auto splits_in = buf_rng.split(8);
    auto splits_out = out_rng.split(8);

    this->e_block->direct(in, buf_rng);
    buf_rng ^= key;

    for (size_t i = 0; 8 > i; i++)
        this->s_blocks[i]->direct(splits_in[i],
                                  splits_out[i]);

    this->p_block->direct(out_rng, out);
}
\end{lstlisting}

\begin{lstlisting}[language=c++, caption={Класс, реализующий получение ключа}]
KeyBlock::KeyBlock(BitRange key,
                   std::shared_ptr<PKeyBlock> p_block,
                   std::vector<std::shared_ptr<ShiftKeyBlock>> shift_blocks,
                   std::shared_ptr<HKeyBlock> h_block)
    : p_block(p_block), shift_blocks(shift_blocks),
      h_block(h_block)
{
    if (64 != key.size())
        throw;

    if (nullptr == p_block || nullptr == h_block)
        throw;

    for (auto block : shift_blocks)
        if (nullptr == block)
            throw;

    char buf[7];
    BitRange inner (buf, 0, 56);

    if (!this->check(key, inner))
        throw;

    this->init(inner);
}

void KeyBlock::get(size_t i, BitRange out)
{
    if (16 <= i)
        throw;

    BitRange tmp (this->keys[i], 0, 48);
    out.copy(tmp);
}

bool KeyBlock::check(BitRange key, BitRange out)
{
    bool k = true;
    char sum = 0;
    auto iterk = key.begin(), itero = out.begin();

    for (size_t i = 0; 8 > i; i++, ++iterk)
    {
        sum = 0;

        for (size_t j = 0; 7 > j; j++, ++itero, ++iterk)
            sum += (*itero = *iterk) ? 1 : 0;

        if (sum % 2 != ((*iterk) ? 1 : 0))
            k = false;
    }

    return k;
}

void KeyBlock::init(BitRange key)
{
    if (56 != key.size())
        throw;

    char buffer[7], tmp[6];
    BitRange buf_rng (buffer, 0, 56), shift_rng (tmp, 0, 28),
             out_rng (tmp, 0, 48);
    auto splits_buf = buf_rng.split(2);
    BitRange c = splits_buf[0], d = splits_buf[1];

    this->p_block->direct(key, c, d);

    for (size_t i = 0; 16 > i; i++)
    {
        this->shift_blocks[i]->direct(c, shift_rng);
        c.copy(shift_rng);
        this->shift_blocks[i]->direct(d, shift_rng);
        d.copy(shift_rng);

        this->h_block->direct(buf_rng, out_rng);

        memmove(this->keys[i], tmp, 6);
    }
}

void KeyBlock::getKey(BitRange in, BitRange out)
{
    if (64 != in.size() || 56 != out.size())
        throw;

    bool valid = true;
    auto iter_in = in.begin(), iter_out = out.begin();

    for (size_t i = 0; valid && 8 > i; i++, ++iter_in)
    {
        int sum = 0;

        for (size_t j = 0;
             valid && 7 > j;
             j++, ++iter_out, ++iter_in)
            sum += (*iter_out = *iter_in) ? 1 : 0;

        if ((*iter_in ? 1 : 0) != sum % 2)
            valid = false;
    }

    if (!valid)
        throw;
}
\end{lstlisting}

Код остальных блоков является выборкой по таблице / массиву и не будет
приведен в отчете.

\subsection{Тестирование}
\begin{table}[h]
\small
\begin{tabular}{|r|l|l|l|}
\hline
№ & Исходные данные                                                                                                            & Ожидаемый результат                                                                                                                                                                           & Фактический результат                                                                                                                                                                                                                                                                                                                                                                                                                                                                                                                \\ \hline
1 & \begin{tabular}[c]{@{}l@{}}abcdefgh\\ ijklmnop\\ qrstuvwx\\ yz\textbackslash{}0a\end{tabular}                              & \begin{tabular}[c]{@{}l@{}}abcdefgh\\ ijklmnop\\ qrstuvwx\\ yz\textbackslash{}x0a\textbackslash{}x00\textbackslash{}x00\textbackslash{}x00\textbackslash{}x00\textbackslash{}x00\end{tabular} & \begin{tabular}[c]{@{}l@{}}abcdefgh\\ ijklmnop\\ qrstuvwx\\ yz\textbackslash{}x0a\textbackslash{}x00\textbackslash{}x00\textbackslash{}x00\textbackslash{}x00\textbackslash{}x00\end{tabular}                                                                                                                                                                                                                                                                                                                                        \\ \hline
2 & Пустой файл                                                                                                                & Пустой файл                                                                                                                                                                                   & Пустой файл                                                                                                                                                                                                                                                                                                                                                                                                                                                                                                                          \\ \hline
3 & \begin{tabular}[c]{@{}l@{}}Попытка расшифровать\\ сообщение из теста №1\\ с другим ключом\\ (во втором блоке)\end{tabular} & \begin{tabular}[c]{@{}l@{}}abcdefgh\\ \textless{}3 произвольных\\ блока по 8 байт\textgreater{}\end{tabular}                                                                                  & \begin{tabular}[c]{@{}l@{}}abcdefgh\\ \textbackslash{}xb7\textbackslash{}xf9\textbackslash{}x12\textbackslash{}x62\textbackslash{}xad\textbackslash{}xec\\ \textbackslash{}xe5\textbackslash{}x78\textbackslash{}xaf\textbackslash{}xe1\textbackslash{}x0a\textbackslash{}x7a\\ \textbackslash{}xb5\textbackslash{}xf4\textbackslash{}xfd\textbackslash{}x70\textbackslash{}xa7\textbackslash{}xe9\\ \textbackslash{}x73\textbackslash{}x0e\textbackslash{}xc0\textbackslash{}x82\textbackslash{}x8a\textbackslash{}x08\end{tabular} \\ \hline
\end{tabular}
\end{table}

